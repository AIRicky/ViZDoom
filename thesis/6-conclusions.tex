\chapter{Conclusions}
	It is strongly believed that Vizia project has been carried out successfully and provides a stable and lightweight tool for research on reinforcement learning in 3-dimensional environment. API is equipped with multiple useful modes (see Section \ref{sec:architecture_modes}) which allow apprenticeship learning, multiplayer game and obviously ordinary learning in the reinforcement learning paradigm. What is more, different modes enable researchers to decide if full control over game engine's processing is needed or if it is the game that should set the pace. Furthermore, the mechanism of scenarios has been employed with much success, making it very easy to create custom reasearch conditions. Section \ref{sec:performance} shows that Vizia's performance is satisfactory and conducted experiment (see Chapter \ref{ch:experiment}) proves that using Vizia is viable in practice. 

\section{Achieved Goals}
	\begin{itemize}
		\item Enforcement of full control over 3D engine processing.
		\item Satisfactory performance and stability.
		\item Custom scenarios support.
		\item Spectator mode allowing apprenticeship learning.
		\item Support for multiplayer.
	\end{itemize}

\section{Future Work}
	\begin{itemize}
		\item Lua binding.
		\item Windows and Mac support.
	\end{itemize}