
\chapter{Introduction}
\label{ch:introduction}
\section{Motivation}
\label{sec:motivation}

Visual signals are one of the main sources of information about the surrounding environment for humans and animals.
While computers have already greatly exceeded humans in terms of raw data processing, they still do not match their ability to process images.
However, recent increase in computing power and emergence of affordable technologies allowing to perform general computations on graphics cards (CUDA, OpenCL) enabled significant progress in this area.
An important part of this advance are deep convolutional neural networks (CNNs).


CNNs for the first time came into prominence when they were used to classify 1.6 millions images into 1000 classes~\cite{NIPS2012_4824}.
This solution was so effective that now this is the prevalent method of image classification.
This success hugely popularized CNNs and soon enough they were used for much more complex tasks.
In combination with reinforcement learning, they were used to model intelligent agent which attempted playing 7 Atari 2600 games from Arcade Learning Platform\cite{mnih-atari-2013}.
In 6 of them the achieved results were better than in all the previous approaches, and in the case of 3 results were better than those achievable by human players.

The main drawback of Atari games is that they are only two-dimensional, so their usability at solving real-world problems is quite limited, but reinforcement learning itself was successfully used for identifying features in static images \cite{conf/cvpr/GoodrichA12} and in robotics for controlling racing slot car using only image from the camera \cite{rieijcnn12} and even for steering RC helicopter\cite{Abbeel07anapplication}.
Bigger scope of practical applications is offered by 3D games which are a better approximation of the real-world.
% <---
Among 3D games, First Person Shooter (FPS) games are particularly interesting because the first-person perspective is a good equivalent of the image obtained from the mobile robot vision systems.
Also important is their great popularity, and simple game rules make it easy to implement rewarding mechanism.
% >---

FPS games have already been successfully used in research on artificial intelligence, especially the most popular ones such as Unreal Tournament \cite{6314567} \cite{6922494}, Counter-Strike \cite{5035619} or Quake III Arena \cite{el2007hybrid}.
However, in these studies agents acted upon high-level information like positions of walls, enemies, locations of item etc, which are normally inaccessible to human player.
Supplying only raw visual information would relieve researchers of the burden of supplying AI high-level information and handcrafted features.
What is more it would force agents to become more autonomous and behave in a way more resembling real intelligent agents (humans and animals).
So far no studies have been conducted on the reinforcement learning from visual information obtained from 3D FPS games.


Currently there are no environments that allow using FPS games for research on artificial intelligence algorithms, in which agents rely exclusively on raw visual information.
This could be a serious factor that is impending the progress of visual information-based research on reinforcement learning.
Engaging in that kind of research would require a large amount of work associated with creation of an environment integrating game engine with interface suited to reinforcement learning paradigm.
The existence of such a tool would allow conducting experiments and focus on the goal of research without having to worry about availability of the test environments.
 

\section{Aims and scope}
%%W rozdziale 1 nie pojawia sie nazwa VIZIA
The main aim of this thesis is, thus Section~\ref{sec:motivation}, to create an easy to use and flexible environment for research on intelligent agents that work and learn using the raw visual information generated by the engine of 3D FPS game. 
% <---
In order to confirm usability of the created environment experiments on simple scenarios should be conducted.
% >---

The goal will be achieved by meeting the following objectives:
\begin{itemize}
 \item to compare and select needed technologies,
 \item to implement the environment,
 \item define API,
 \item to define and implement test scenarios in the created environment,
 \item to implement learning algorithms for deep neural networks,
 \item to conduct experiments for simple test scenarios.
\end{itemize}
% <---
It was assumed that VIZIA environment has to meet the following assumptions:


\begin{enumerate}
\item based on popular open-source 3D FPS game (ability to modify the code and the freedom of publication)
\item lightweight (portability and the ability to run multiple instances)
\item fast (AI should be a limiting factor for the speed of processing, not the game itself)
\item total control over game's processing (pausing processing for the duration of AI run)
\item customizable resolution and rendering parameters (adjusting the image generated by the game to meet the needs)
\item multiplayer (testing agent in the fight against the human player or other AI)
\item spectator mode (agent learning from observations of a human playing)
\item easy and convenient creation of custom scenarios (creating test environments tailored to the needs)
\item reinforcement learning friendly API in C++ (use for the research on reinforcement learning; the ability to bind other languages)
\item Python bindings (popularity of Python in machine learning)
\item multiplatform (availability of full functionality on Linux, Windows and Mac OS)
\end{enumerate}

% >---

%The created environment should be based on open-source, lightweight 3D FPS game that allows total control over game's processing and customising resolution and rendering parameters.
%There should be implemented spectator mode, in which agent can learn from observations of a human playing.
%Support for easy and convenient creation of custom scenarios should be essential part of the environment. 
%Designed API should be reinforcement learning friendly and implemented in C++ with bindings to Python and possibly other languages (Java, Lua etc.)
%The environment should be multiplatform, focused on Linux, with Windows and OS X support.

%Brakuje o multiplayer i o tym, ze ma byc szybkie (powod).
%To jest bardzo wazny paragraf i chcialbym, zeby go rozwinac. Niech bedzie sie zaczynal od tego, ze przyjeto, ze srodowisko ma spelniac nastepujace zalozenia. I tutaj punktowana lista zalozen: 1) based on open-source (powod), 2) lightweight (powod)...	
	
\section{Thesis organization}


This thesis is structured as follows. 
Chapter~\ref{ch:architecture} gives an overview of technologies and tools used to develop the VIZIA environment and describes environment's architecture. It addresses design decisions and problems and contains result of performance tests. 
Chapter~\ref{ch:api} presents the designed application programming interface, python binding and shows API's usage examples. 
Chapter~\ref{ch:scenarios} gives definition of scenario and presents tools and methods for creating scenarios. It contains the description of designed scenarios. 
Chapter~\ref{ch:experiment} shows methods used for conducting experiments and their results. 
Chapter~\ref{ch:conclusions} concludes this thesis and proposes directions for future work.

\section{Contributions}
	\subsection{Engineering Project}
	\begin{description}
		\item[Michał Kempka] \hfill
			\begin{itemize}
				\item part of interface design,
				\item testing and experiments,
				\item part of python binding,
				\item python examples,
				\item scenarios creation,
				\item support for configuration files.
			\end{itemize}
		\item[Grzegorz Runc] \hfill
			\begin{itemize}
				\item generation of depth buffer,
				\item support for off-screen rendering.
			\end{itemize}
		\item[Jakub Toczek] \hfill
			\begin{itemize}
				\item python binding,
				\item java binding.
			\end{itemize}
		\item[Marek Wydmuch] \hfill
			\begin{itemize}
				\item VIZIA API design,
				\item VIZIA API implementation,
				\item VIZIA Module in Doom engine implementation,
				\item CMake configuration.
			\end{itemize}
	\end{description}
	
   	
	\subsection{Thesis}
	\begin{description}
		\item[Michał Kempka] \hfill
			\begin{itemize}
				\item foundations for Chapter \ref{ch:api},
				\item Chapter \ref{ch:scenarios},
				\item Chapter \ref{ch:experiment},
				\item Chapter \ref{ch:conclusions},
				\item translation of the abstract.
			\end{itemize}
		\item[Grzegorz Runc] \hfill
			\begin{itemize}
				\item Chapter~\ref{ch:introduction},
				\item Section~\ref{sec:technologies}.
                \item the abstract
			\end{itemize}
		\item[Jakub Toczek] \hfill
			\begin{itemize}
				\item Chapter~\ref{ch:api}.
			\end{itemize}
		\item[Marek Wydmuch] \hfill
			\begin{itemize}
				\item Section~\ref{sec:architecture},
				\item Section~\ref{sec:architecture_solutions},
				\item Section~\ref{sec:performance},
			\end{itemize}
	\end{description}
