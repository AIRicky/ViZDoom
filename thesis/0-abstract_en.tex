\chapter*{Abstract}

This thesis is concentrated on usage of 3-dimensional FPS games in research on intelligent agents that act and learn based on purely visual information.


In the last couple of years computers significantly exceeded humans in terms of raw data processing.
It does not apply, however, to visual information processing, which has been equally successful only for certain applications.
Convolutional Neural Networks are successfully used for image classification, and combined with reinforcement learning they prove to be useful in robotics, among others.
Deep reinforcement learning has also been employed to teach an intelligent agent to play 7 Atari 2600 games.
However, range of practical applications of such technology is very limited, as it works with visual input which is two-dimensional.
Adding the depth, thus involving games in three-dimensional environments, would notably increase real-life practicality.
Due to first-person perspective, First-person shooter games (FPS) are especially interesting. However, no research has been conducted so far, that involves reinforcement learning from raw visual data generated by an FPS game.

The main aim of this thesis was to create an easy-to-use environment for research on intelligent agents that act and learn based on purely visual data generated by a three-dimensional FPS game, and conducting simple experiments for the most basic scenarios.


Work on this thesis resulted in creation of environment employing a vintage game -- Doom to expose an interface proper for reinforcement learning -- VIZIA. VIZIA's application programming interface was fully written in C++ and allows to make in-game actions, retrieve game's screen buffer and plenty of in-game parameters such as player's health or ammunition. 
The API offers a myriad of configuration options which can easily be written and stored in text files.
In addition to C++ support bindings for Python and Java has been created as these languages are more popular for AI research purposes. 
The environment offers a mechanism of scenarios that enables researchers to design custom research conditions that support reinforcement learning paradigm.
What is more, multiple different modes of operation are available and allow to take full control over game's engine processing, perform apprenticeship learning or even engage in multiplayer/multiagent skirmishes. 

VIZIA is mainly targeted at Linux platform and provides options for off-screen rendering that requires no graphical environment, therefore can be used with remote terminals.
The environment allows to reach processing speeds counted in thousands of frames per second.

In order to test practicality of using VIZIA in AI research, a simple experiment has been conducted that in fact proves that the environment serves its purpose.
