
\chapter{Framework Outline}

\section{Used Technologies}
What we used and why.
\begin{itemize}
\item zdoom, mention alternatives
\item linux focus, cpp core, python wrapper
\item acs scripting in doombuilder 2 for scenarios
\item python and lasagne for experiments
\end{itemize}

\section{Architecture}
Nice diagram (in DOOM style) with the arcitecture.
\begin{itemize}
\item Zdoom separate process.
\item Boost interprocess: shared memory to comunicate with zdoom.
\item Flow control and PLAYER vs SPECTATOR mode.
\item Warnings and exceptions.
\end{itemize}

\section{Problems and Solutions}
\begin{itemize}
\item Why shared memory and separate doom process and what it entails.
\item Why make/set action are like they are. Why action is a vector not just number.
\item Why state is copied in Python but not in cpp.
\item Zbuffer struggles.
\item Why Windows and Mac are not supported so well.
\item Why scenario is effectively divided into config file nad doom iwad file.
\item Why multiplayer is barely usable.
\end{itemize}

\section{Performance}
Table with some fps ratings and a graph.
Conclusions: it's fast enough, any reasonably good AI will be much slower during learning process.

\section{Building process}
\subsection{Prerequisites}
\begin{itemize}
\item preferably linux
\item cmake
\item make
\item gcc 4.??
\item boost v ?
\item python 2.6 with numpy (v?) for pyhon wrapper
\item java ? for java wrapper
\item ...
\end{itemize}
\subsection{Compilation}
cmake, make and they lived happily ever after

