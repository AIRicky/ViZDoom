\chapter*{Streszczenie}

Niniejsza praca jest poświęcona wykorzystaniu trójwymiarowych gier typu FPS (ang. first-person shooter) do badań nad inteligentnymi agentami, które działają i uczą się w oparciu o informację obrazową.

W ciągu ostatnich lat komputery znacznie przewyższyły ludzi pod względem możliwości prze-twarzania surowych danych.
Nie dotyczy to jednak przetwarzania informacji obrazowej, w przypadku której postęp dotyczy tylko ograniczonych zastosowań.
Konwolucyjne sieci neuronowe z powodzeniem są wykorzystywane do klasyfikacji zdjęć, a w połączeniu z uczeniem ze wzmocnieniem znajdują zastosowanie m.in. w robotyce.
Głębokie uczenie ze wzmocnieniem zostało również wykorzystane do nauczenia agenta grania w 7 gier z Atari 2600.
Przestrzeń praktycznych zastosowań jest jednak w tym przypadku ograniczona, ponieważ jest to informacja obrazowa o dwuwymiarowym charakterze.
Dodanie aspektu głębi, a zatem wykorzystanie gier trójwymiarowych, znacznie zwiększyłoby możliwość praktycznych zastosowań.
Ze względu na obserwowanie świata z perspektywy pierwszej osoby szczególnie interesujące pod tym względem wydają się być gry FPS, jednak dotychczas nie prowadzono badań nad wykorzystaniem uczenia ze wzmocnieniem z informacji obrazowej pochodzącej z gry FPS.

Celem pracy było opracowanie łatwego w użyciu środowiska do badań nad inteligentnymi agentami, które działają i uczą się w oparciu o informację obrazową generowaną przez trójwymiarową grę FPS, oraz przeprowadzenie eksperymentów dla prostych scenariuszy.

Rezultatem niniejszej pracy jest środowisko VIZIA oparte na grze Doom dostosowane do potrzeb paradygmatu uczenia ze wzmocnieniem.
Stworzony został interfejs programistyczny zaimplementowany w C++ pozwalający na wykonywanie akcji, pobieranie obrazu i informacji o bohaterze gry, jak i swobodny dobór parametrów wykonania gry.
Parametry te mogą być zapisywane w postaci plików konfiguracyjnych.
Zapewniono również wsparcie dla Pythona i Javy.
Środowisko to wspiera tworzenie scenariuszy testowych w zewnętrznych edytorach i implementację funkcji nagrody za pomocą języka skryptowego Action Code Script.
Poza trybem pełnej kontroli nad grą zaimplementowano również tryb obserwatora, pozwalający na uczenie agenta na podstawie rozgrywki prowadzonej przez człowieka.
Dodatkowo środowisko może pracować w trybie zarówno synchronicznym, w którym gra czeka na akcje agenta, jak i asynchronicznym, w którym gra działa ze stałą prędkością, co pozwala na wykorzystanie jej w rozgrywkach sieciowych.

Stworzone oprogramowanie zostało przygotowane do pracy w systemie Linux i udostępnia tryb niewymagający środowiska graficznego, dzięki czemu może być uruchamiane na zdalnych stanowiskach.
Środowisko umożliwia przetwarzanie z prędkościami liczonymi w tysiącach klatek na sekundę.

Przeprowadzone dla prostych środowisk testowych eksperymenty wykorzystujące uczenie ze wzmocnieniem głębokich sieci neuronowych wskazują na użyteczność stworzonego środowiska do badań nad głębokim uczeniem ze wzmocnieniem z informacji obrazowej.
	
